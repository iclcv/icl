%% the document type and settins
\documentclass[a4paper,fontsize=10pt,dvipsnames]{scrreprt}

%% some packages for utf8 characters, german umlauts, nice typesetting
\usepackage[utf8x]{inputenc}
\usepackage[ngerman]{babel}
\usepackage[T1]{fontenc}
\usepackage{microtype}
\usepackage{lmodern}

%% packages for graphic visualization floats and subfloats
\usepackage{graphicx}
\usepackage{geometry}
\usepackage{float}
\usepackage{subfigure}
\usepackage[labelfont={bf,sf,small}, font=small, format=plain]{caption}

%% to create plots within latex document
\usepackage{pgfplots}

%% math packages
\usepackage{mathtools}
\usepackage{amsfonts}

%% \nameref to make references to names of labels
\usepackage{nameref}

%% to change space between lines
\usepackage{setspace}

%% to show source code in document
\usepackage{listings}

%% acronym package for acronyms in text
\usepackage[nolist,nohyperlinks]{acronym}

%% a nice way to use todos in document
\usepackage{todonotes}
%\usepackage[disable]{todonotes}

%% hyperref makes links in document, the hypersetup makes them invisible
\usepackage{hyperref}
\hypersetup{
    unicode=false,          % non-Latin characters in Acrobat’s bookmarks
    pdftoolbar=true,        % show Acrobat’s toolbar?
    pdfmenubar=true,        % show Acrobat’s menu?
    pdffitwindow=false,     % window fit to page when opened
    pdfstartview={FitH},    % fits the width of the page to the window
    pdftitle={My title},    % title
    pdfauthor={Author},     % author
    pdfsubject={Subject},   % subject of the document
    pdfcreator={Creator},   % creator of the document
    pdfproducer={Producer}, % producer of the document
    pdfkeywords={keyword1} {key2} {key3}, % list of keywords
    pdfnewwindow=true,      % links in new window
    colorlinks=true,       % false: boxed links; true: colored links
    linkcolor=black,          % color of internal links
    citecolor=black,        % color of links to bibliography
    filecolor=black,      % color of file links
    urlcolor=black           % color of external links
}

\definecolor{lightgray}{cmyk}{0,0,0,0.08}
\lstloadlanguages{XML}
\lstset{
    language=XML,                   % the language of the code
    basicstyle=\footnotesize,       % the size of the fonts that are used for the code
    numbers=left,                   % where to put the line-numbers
    numberstyle=\footnotesize,      % the size of the fonts that are used for the line-numbers
    stepnumber=2,                   % the step between two line-numbers. If it's 1, each line will be numbered
    numbersep=5pt,                  % how far the line-numbers are from the code
    backgroundcolor=\color{lightgray},  % choose the background color. You must add \usepackage{color}
    showspaces=false,               % show spaces adding particular underscores
    showstringspaces=false,         % underline spaces within strings
    showtabs=false,                 % show tabs within strings adding particular underscores
    frame=single,                   % adds a frame around the code
    tabsize=2,                      % sets default tabsize to 2 spaces
    captionpos=b,                   % sets the caption-position to bottom
    breaklines=true,                % sets automatic line breaking
    breakatwhitespace=false,        % sets if automatic breaks should only happen at whitespace
    title=\lstname,                 % show the filename of files included with \lstinputlisting; also try caption instead of title
    escapeinside={\%*}{*)},         % if you want to add a comment within your code
    morekeywords={*,...}            % if you want to add more keywords to the set
}

\begin{acronym}
    \acro{ICL}{\emph{Image Component Library}}
\end{acronym}


%% the bibiography style for \cite 
%\bibliographystyle{plain}
%\bibliographystyle{abbrv}
%\bibliographystyle{alpha}
\bibliographystyle{apalike}
%\bibliographystyle{unsrt}


%%%%%%%%%%%%%%%%%%%%%%%%%%%%%%%%%%%%%%%%%%%%%%%%%%%%%%%%%%%%%%%%%%%%%%%%
%                      here starts the document                        %
%%%%%%%%%%%%%%%%%%%%%%%%%%%%%%%%%%%%%%%%%%%%%%%%%%%%%%%%%%%%%%%%%%%%%%%%
\begin{document}

%% set document title and automatically create title page
\title{Project report ICL-Project}
\date{\today}
\author{John Doe\\ Technische Fakult\"at, Universit\"at Bielefeld, 
        \and Richard Row, \\ Technische Fakult\"at, Universit\"at Bielefeld}
\maketitle

%% add table of contents
%\thispagestyle{empty}\ \clearpage
\setcounter{page}{0}
\pagenumbering{roman}
\thispagestyle{empty}
\tableofcontents
\clearpage

%% list of todos 
\listoftodos
\clearpage

\setcounter{page}{0}
\pagenumbering{arabic}

% include subdocuments
%
% chapter introduction
%

This document is intended as a (more or less short) tutorial for the computer vision library \textbf{ICL} (\textbf{I}mage \textbf{C}omponent \textbf{L}ibrary). It will explain different aspects, that are useful to work with the ICL. The following outline shall give a short but documented overview over this document:

\begin{enumerate}
\item \textbf{What is the ICL:}\\ Here the underlying goals and ideas of the ICL will be presented.
\item \textbf{Feature Overview:}\\ The ICL provides a large set of features, packages, algorithms and utility classes. To motivate the reader for reading also the following chapters, the \emph{highlights} of the ICL are presented here.
\item \textbf{What is an Image:}\\ An image representation is fundamentally for the development of high performance computer vision algorihtms. It will be explained how the image representation is implemented in C++ as base class for ICL images. In particular it will be shown why we use template classes in combination with inheritance for the image class and, what problems arise therewith. 
\item \textbf{The Image class:}\\ Some functionalities are implemented as member functions, however most others are not. Here the ICL's image class interface is discussed.
\item \textbf{C++ Templates:}\\ To make later chapters more accessible, C++ template techniques are explained in detail. In particular, it will be shown, how templates are translated by the compiler, and how they can be used \small{\textbf{i)}} To reduce the amount of redundant source code and \small{\textbf{ii)}} To accelerate code without using constants for each possible parameter of functions.    
\item \textbf{Simple Image processing:}\\ How can simple Image processing algorithms be implemented (and how can they implemented elegantly) using the ICL.
\item \textbf{Overview over ICL Packages:}\\ The ICL consists of a set of (currently) 10 sub-packages that more or less base on each other. In this part, the contents and the basic ideas of these packages are introduced and explained. 
\item \textbf{Common ICL-classes and ICL-Interfaces:}\\ In many packages, interfaces (such as e.g. a \inlinecode{Grabber} for images -- a image source) are defined that influenced the design of the whole library heavily. These interfaces and also a variety of other very common classes and class sets are presented and explained.
\item \textbf{Graphical User Interfaces (GUI):}\\ The ICLQt package provides a powerful wrapper for Qt-based applications. In this part it will be shown, how simple and even more complex GUI's can be created and especially, how user input can be synchronized with the applications worker thread.
\item \textbf{Writing advanced applications:}\\ Here some advanced programming techniques are exemplified: Writing applications with GUI support, managing several threads, handling program arguments and much more.
\item \textbf{ICL-Development:}\\ICL developers need some deeper insights into the file structure and the makefile system. These information will be given in this part.
\item \textbf{ICL-Projects:}\\Development of ICL-bases applications can be performed very conveniently using a special makefile system and directory structure provided by the so called \emph{ICLProjects} svn branch. How to get these projects, how to add own projects here and how to include other projects and external libraries is shown in this (currently) last part. 
\end{enumerate} 


The following chapter will examine core features and design principles of the ICL.





%% bibliography
\newpage
\phantomsection
\addcontentsline{toc}{chapter}{Bibliograpie}
\bibliography{../library}

%% appendix
\newpage
\appendix
%% include subdocuments for appendix
\include{sections/appendix}

\end{document}
